\documentclass{beamer}

% Required packages 
\usepackage{fontspec}
\usepackage{fontawesome}
\usepackage{hyperref}
%\usepackage[ngerman]{babel}
\usetheme{metropolis}           

\newcommand{\cw}{47}

% Header
\title{AMSE - Kalendarwoche \cw{}}
\date{20. November 2019}
\author{Benjamin Fischer}
\institute{benjamin.f.fischer@fau.de}

% Document
\begin{document}
  \maketitle

  % Structure (always the same)
  \begin{frame}
    \begin{itemize}
      \item Zusammenfassung Kalendarwoche \cw{}
      \begin{itemize}
        \item Erreichte Ziele
        \item Aufgetretene Probleme
      \end{itemize}
      \item Ausblick nächste Woche
    \end{itemize}
  \end{frame}

  \begin{frame}
    \frametitle{Zusammenfassung Kalendarwoche \cw{}}
    \begin{itemize}
      \item Quelle von \texttt{abgeordnetenwatch.de} funktioniert mit ODS
      \item Backend funktioniert inzwischen größtenteils
      \begin{itemize}
        \item 3 Threads (REST-API, Link-Scraper, Protokoll-Processor)
        \item Kommunikation über Datenbank und Semaphoren
        \item \texttt{DEMO}
      \end{itemize}
    \end{itemize}
  \end{frame}

  \begin{frame}
    \frametitle{Ausblick nächste Woche}
    \begin{itemize}
      \item Arbeit am Backend vorerst abschließen
      \begin{itemize}
        \item Refactoring
        \item Dokumentation
        \item Tests
      \end{itemize}
      \item optional: docker-compose für Frontend - Backend - ODS
    \end{itemize}
  \end{frame}

\end{document}
