\documentclass{beamer}

% Required packages 
\usepackage{fontspec}
\usepackage{fontawesome}
\usepackage{hyperref}
%\usepackage[ngerman]{babel}
\usetheme{metropolis}           

\newcommand{\cw}{49}

% Header
\title{AMSE - Kalendarwoche \cw{}}
\date{04. Dezember 2019}
\author{Benjamin Fischer}
\institute{benjamin.f.fischer@fau.de}

% Document
\begin{document}
  \maketitle

  % Structure (always the same)
  \begin{frame}
    \begin{itemize}
      \item Zusammenfassung Kalendarwoche \cw{}
      \begin{itemize}
        \item Erreichte Ziele
        \item Aufgetretene Probleme
      \end{itemize}
      \item Ausblick nächste Woche
    \end{itemize}
  \end{frame}

  \begin{frame}
    \frametitle{Zusammenfassung Kalendarwoche \cw{} (Backend)}
    \begin{itemize}
      \item Arbeit an Backend vorerst abgeschlossen
      \item Backend stellt vier Routes zur Verfügung
      \begin{itemize}
        \item \texttt{home}: Allgemeine Informationen über API
        \item \texttt{info}: Allgemeine Informationen über Daten
        \item \texttt{deputies}: Liste der Abgeordneten des Bundestags
        \item \texttt{profile/\{name\}}: Profil des Arbgeordneten \texttt{name}
      \end{itemize}
      \item Wrapper für ODS ($\rightarrow$ Route \texttt{deputies})
      \begin{itemize}
        \item Wenn ODS erreichbar, werden Daten von ihm bezogen
        \item Ansonsten Fallback auf \texttt{abgeordnetenwatch.de}
      \end{itemize}
      \item Linting \& Unittests
    \end{itemize}
  \end{frame}

  \begin{frame}
    \frametitle{Zusammenfassung Kalendarwoche \cw{} (Backend) (2)}
    \begin{itemize}
      \item Test-Skripte für folgende Zwecke:
      \begin{itemize}
        \item Parsen der aktuell verfügbaren Protokolle
        \item Linten des Quellcodes innerhalb des Backends
      \end{itemize}
      \item Sentiment-Analyse für die deutsche Sprache unvollständig
      \begin{itemize}
        \item Verwendete Bibliothek: \texttt{textblod\_de}
        \item Nur wenige Wörter besitzen passende \textit{subjectivity}
        \item Wesentlich besseres Ergebnis mit Übersetzung ins Englische
        \item Benutzt intern Google NLP API (Limit: 1000 Wörter/Tag)
        \item In Zukunft vielleicht mit anderer Library übersetzen
      \end{itemize}
    \end{itemize}
  \end{frame}

  \begin{frame}
    \frametitle{Zusammenfassung Kalendarwoche \cw{} (Frontend)}
    \begin{itemize}
      \item Entwicklung hauptsächlich innerhalb des WebViews
      \begin{itemize}
        \item einfacher \& schneller
        \item zum Testen aber auch über \texttt{adb} und Smartphone möglich
      \end{itemize}
      \item Problem hierbei: \texttt{Same Origin Policy} des Browsers
      \begin{itemize}
        \item[\tiny\faClose] CORS (Server) aktivieren
        \item[\tiny\faClose] Entwicklung nur auf Device
        \item[\tiny\faClose] Verwendung von Proxies
        \item[\tiny\faCheck] Für WebView Chromium mit deaktivierter \texttt{SOP}
                          verwenden 
      \end{itemize}
      \item Frontend im Gegensatz zu Backend auf Host (kein Docker)
      \begin{itemize}
        \item eher mehr Probleme durch \texttt{adb}, \texttt{volumes} und
                Netzwerke
        \item Expo-Client stellt bereits \textit{Hot-Reloading} zur Verfügung
      \end{itemize}
      \item Frontend und Backend über Makefile getrennt
    \end{itemize}
  \end{frame}

  \begin{frame}
    \frametitle{Ausblick nächste Woche}
    \begin{itemize}
      \item In React Native einarbeiten (hauptsächlich)
      \item Nach Möglichkeiten für Unittests \& Linting suchen
      \item Einfaches Layout für Applikation überlegen
    \end{itemize}
  \end{frame}

\end{document}
